\chapter*{ВВЕДЕНИЕ}

Проблема сжатия данных является актуальной уже на протяжении длительного времени.
С постоянным ростом объема данных, эффективное сжатие становится необходимостью для экономии пространства хранения и более быстрой передачи информации.
Вопрос сжатия изображений является одним из актуальных.

Сжатие изображений является процессом уменьшения размера изображения.
Оно позволяет сократить объем данных, улучшить скорость передачи и сэкономить место на устройствах хранения.
Общим свойством алгоритмов сжатия можно считать то, что они выполняются над всем изображением.\cite{adaptive_method_image_compression}

Таким образом, сжатие изображений является важным аспектом обработки данных, который позволяет сократить объем информации и улучшить скорость передачи.

Целью данной работы является изучение и описание известных алгоритмов сжатия изображений.

Для достижения поставленной цели необходимо выполнить следующие задачи:
\begin{itemize}
    \item изучить принципы сжатия данных и понять, как они применяется к изображениям;
    \item классифицировать методы сжатия изображений;
    \item описать рассматриваемые алгоритмы на формальном языке;
    \item выделить преимущества и недостатки рассматриваемых алгоритмов;
    \item сравнить рассматриваемые алгоритмы.
\end{itemize}

\addcontentsline{toc}{chapter}{ВВЕДЕНИЕ}